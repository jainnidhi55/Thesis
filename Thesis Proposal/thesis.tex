%for a more compact document, add the option openany to avoid
%starting all chapters on odd numbered pages
\documentclass[hidelinks, 12pt]{cmuthesis}

% This is a template for a CMU thesis.  It is 18 pages without any content :-)
% The source for this is pulled from a variety of sources and people.
% Here's a partial list of people who may or may have not contributed:
%
%        bnoble   = Brian Noble
%        caruana  = Rich Caruana
%        colohan  = Chris Colohan
%        jab      = Justin Boyan
%        josullvn = Joseph O'Sullivan
%        jrs      = Jonathan Shewchuk
%        kosak    = Corey Kosak
%        mjz      = Matt Zekauskas (mattz@cs)
%        pdinda   = Peter Dinda
%        pfr      = Patrick Riley
%        dkoes = David Koes (me)

% My main contribution is putting everything into a single class files and small
% template since I prefer this to some complicated sprawling directory tree with
% makefiles.

% some useful packages
\usepackage{times}
\usepackage{fullpage}
\usepackage{graphicx}
\usepackage{amsmath}
\usepackage{subfiles}
\usepackage[numbers,sort]{natbib}
\usepackage[backref,pageanchor=true,plainpages=false, pdfpagelabels, bookmarks,bookmarksnumbered,
%pdfborder=0 0 0,  %removes outlines around hyper links in online display
]{hyperref}
%% \usepackage{subfigure}

\usepackage{amsmath}
\usepackage{amssymb}
\usepackage{gensymb} 
\usepackage{dsfont}
\usepackage{graphicx}
\usepackage{float}
\let\labelindent\relax
\usepackage{enumitem}
\usepackage{hyperref}
\usepackage{tabu}
\usepackage{array}
\usepackage{caption}
\usepackage{subcaption}
\usepackage{colortbl}
\usepackage{tikz}
\usepackage{algorithm}
% \usepackage{algorithmic}
\usepackage{algpseudocode}
\usepackage{xparse}
\usetikzlibrary{positioning}


% \usepackage[utf8]{inputenc} % allow utf-8 input
\usepackage[T1]{fontenc}    % use 8-bit T1 fonts
% \usepackage{url}            % simple URL typesetting
\usepackage{booktabs}       % professional-quality tables
% \usepackage{amsfonts}       % blackboard math symbols
% \usepackage{nicefrac}       % compact symbols for 1/2, etc.
% \usepackage{microtype}      % microtypography
% \usepackage{}
\usepackage{multirow}
% \usepackage{pbox}
% \usepackage{enumitem}
% \usepackage{bbm}
% \usepackage{unicode-math}
% % \usepackage[table,xcdraw]{xcolor}
% \usepackage{xfrac}
% \usepackage{pifont}% http://ctan.org/pkg/pifont
% \usepackage{dblfloatfix}

% % For figures
% \graphicspath{{figs/}}

\usepackage[prependcaption,textsize=tiny]{todonotes}
\usepackage{setspace}




\DeclareDocumentCommand{\particles}{ O{} O{} O{}}{^{#2}X_{#1}^{#3}}
% \particle[time][index][section]
\DeclareDocumentCommand{\particle}{ O{} O{} O{}}{^{#2}x_{#1}^{#3}}
\newcommand{\state}{x}

\newcommand{\sampled}{\tilde}
\newcommand{\subcap}[1]{\textbf{(\subref{#1})}}
\newcommand{\maction}{\mathcal{M}}
%% \newcommand{\particles}{\mathcal{P}}
%% \newcommand{\particle}{x}
\newcommand{\bin}{b}
\newcommand{\groups}{L}
\newcommand{\measurement}{m}
\newcommand{\measurementSet}{M}
\newcommand{\totalWeight}{\mathcal{W}}
\newcommand{\feature}{\mathcal{S}}
\renewcommand{\algorithmicrequire}{\textbf{Input:}}
\renewcommand{\algorithmicensure}{\textbf{Output:}}
\def\BState{\State\hskip-\ALG@thistlm}

% Approximately 1" margins, more space on binding side
%\usepackage[letterpaper,twoside,vscale=.8,hscale=.75,nomarginpar]{geometry}
%for general printing (not binding)
\usepackage[letterpaper,twoside,vscale=.8,hscale=.75,nomarginpar,hmarginratio=1:1]{geometry}

% Provides a draft mark at the top of the document. 
\draftstamp{\today}{DRAFT}

\begin {document} 
\frontmatter

%initialize page style, so contents come out right (see bot) -mjz
\pagestyle{empty}

\title{ %% {\it \huge Thesis Proposal}\\
{\bf Investigating and Identifying Food Selective Regions in the Brain [DRAFT]}}
\author{Nidhi Jain}
\date{May 11}
\Year{2022}
\trnumber{CMU-CS-22-108}

\committee{
Leila Wehbe, Chair \\
Michael J. Tarr
}

\support{}
\disclaimer{}
% STRETCH
\setstretch{1.1}

% copyright notice generated automatically from Year and author.
% permission added if \permission{} given.

%% \keywords{}

\maketitle

%% \begin{dedication}

%% \end{dedication}

\pagestyle{plain} % for toc, was empty

%% Obviously, it's probably a good idea to break the various sections of your thesis
%% into different files and input them into this file...

\begin{abstract}
  Existing research has demonstrated functional selectivity in the brain for high level categories of faces, places, bodies, and sensory and motor processes. Food, despite its abundance and lack of visual coherence, has not been considered as a category for which there is a visual selective region. In this paper, we investigate responsiveness to food in a large scale natural setting via several statistical methods performed on high-resolution fMRI response dataset to natural scenes. We identify regions consistent across all participants in the high level visual cortex that appear to be functionally selective for food. 
\end{abstract}

\begin{acknowledgments}
    TODO

  
\end{acknowledgments}



\tableofcontents
\listoffigures
% \listoftables

\mainmatter

%% Double space document for easy review:
%\renewcommand{\baselinestretch}{1.66}\normalsize

% The other requirements Catherine has:
%
%  - avoid large margins.  She wants the thesis to use fewer pages, 
%    especially if it requires colour printing.
%
%  - The thesis should be formatted for double-sided printing.  This
%    means that all chapters, acknowledgements, table of contents, etc.
%    should start on odd numbered (right facing) pages.
%
%  - You need to use the department standard tech report title page.  I
%    have tried to ensure that the title page here conforms to this
%    standard.
%
%  - Use a nice serif font, such as Times Roman.  Sans serif looks bad.
%
% Other than that, just make it look good...


\chapter{Introduction}\label{chap:intro}
\subfile{Introduction/intro.tex}

\newpage

\chapter{Background} \label{chap:related_work}
\subfile{Background/background.tex}

\chapter{Methods} \label{chap:work1}
\subfile{methods/methods.tex}

\chapter{Results} \label{chap:work2}
\subfile{results/results.tex}

% \chapter{Work 3} \label{chap:work3}
% \subfile{work3/work3.tex}

\chapter{Conclusion and Future Work}
\subfile{Conclusion/conclusion.tex}

%\appendix
%\include{appendix}

\backmatter

%\renewcommand{\baselinestretch}{1.0}\normalsize

% By default \bibsection is \chapter*, but we really want this to show
% up in the table of contents and pdf bookmarks.
\renewcommand{\bibsection}{\chapter{\bibname}}
%\newcommand{\bibpreamble}{This text goes between the ``Bibliography''
%  header and the actual list of references}
\bibliographystyle{unsrtnat}
\bibliography{ref} %your bib file
\nocite{*}

\end{document}
