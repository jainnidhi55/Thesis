\documentclass[../thesis.tex]{subfiles}

\begin{document}

How are knowledge representations organized in the human brain? Within the visual system, one of the hallmark discoveries of the past several decades has been \textit{category selectivity}~\cite{Sergent1992,Kanwisher1997,Epstein1998,Downing2001} We saw here results that clearly suggest selectivity for food. The code for the food representation is not yet clear. Due to the variability of the food images, their location, image perspective, and most importantly that they recruit areas outside primary visual cortex, it is probable that this is not due to low level features. The lack of consistency between image embedding clusters and voxel activity clusters also suggests that the proposed regions are not due to raw input features. \\

Attention and/or saliency is not a convincing hypothesis on its own because other categories such as faces also have both attention and saliency, and yet there are shown to recruit a subset of the ventral temporal cortex. These results seem to reliably show there are some regions which are especially activated by images pertaining to food. Future work can use other modalities like MEG to see if there is a modulation of these regions from higher level areas. As they are located in the high level visual system, it is possible at least some of their function is feed-forward or automatic. Further consideration of the effect of individual preferences would help better understand the function of these regions.

\end{document}
