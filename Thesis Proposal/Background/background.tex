\documentclass[../thesis.tex]{subfiles}

\begin{document}

The representation of high-level visual information in the human brain has been marked by the phenomenon of \textit{category selectivity}. There are multiple brain regions that show preferential neural responses to specific visual categories: faces, bodies, and places~\cite{Sergent1992,Kanwisher1997,Epstein1998,Downing2001}. Independent of any particular view for \textit{how} these functional brain regions arise~\cite{Kanwisher2000,Tarr2000}, we can agree that they exist because the categories represented are highly relevant for day-to-day behavior. In a similar vein, food is a category that we would expect to be evolutionarily relevant -- being more ancient than social interaction and, arguably, more critical to survival. Curiously, food has not been identified as a visual category for which localized preferential neural responses have been observed. \\

Visual stimulation by food is thoroughly involved, eliciting both cognitive and emotional responses\cite{ouwehand2010eat}. Specifically, the sight of food has been shown to play a significant role in the complex systems of both food selection and regulating food intake \cite{VanderLaan, van2006evolved}. These complex effects emphasize the potential of investigating potentially food selective food regions. \\

Existing work has discovered regions consistently activated by food specifically in the occipital cortex and insula when focusing on visual stimuli of isolated foods compared to non-food items. Some work has found the lateral orbitofrontal cortex to be especially activated by individual images of food as well \cite{VanderLaan}. Most existing work has focused on visual stimuli of isolated food items or other objects. \\

By focusing on isolated food and non-food posed visual stimuli, studies have failed to consider food in a large scale naturalistic setting. Plating practices in restaurants and the abundance of food images on social media are some of the many proofs of how important the surrounding visual aspects of food are. Scientists have also mostly focused on the emotional responses to food, and have not focused on the visual processes. Also, because of averaging across subjects, studies have not found distinct regions for foods and other salient categories. \\

We consider this naturalistic setting, and find two distinct regions in the high level visual cortex that are especially responsive to food, highlighting the importance of this category in the visual system. We name this region to be food-selective, and define this as a region that has significantly high response activity to food than other regions. \\

We also take note that food is particularly challenging because there appears to be less visual coherence across the category. This leads us to question the degree to which top-down semantic knowledge influences any category-selective responses. 

\end{document}
